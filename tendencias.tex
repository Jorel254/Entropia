\documentclass[10pt,a4paper]{article}
\usepackage[utf8]{inputenc}
\usepackage{amsmath}
\usepackage{amsfonts}
\usepackage{amssymb}
\usepackage{graphicx}
\usepackage{float}
\usepackage{multirow}
\usepackage{array}
\usepackage{ragged2e}
\begin{document}

\newpage \normalsize
\section{NETWORKING}
\subsection{Hybrid WAN}
\justifying
Hybrid WAN is a way of connecting two geographically separated WANs to a branch office with the traffic being sent over two different types of connections. One connection is the traditional MPLS that connects to the data center. The other connection is made through a broadband connection to the Internet or as a VPN connection to the data center. All the normal business traffic that is intended to go to the data center takes the MPLS route.
\subsection{Wireless Docking}
\justifying
A docking station that connects to a laptop without a cable. Instead of plugging a laptop into a docking station via USB or a proprietary connector, the laptop transmits via Wi-Fi or WiGig to a dock that is cabled to monitors, printers and external storage drives
\subsection{Software Defined Networks}
\justifying
Technology that separates the control plane management of network devices from the underlying data plane that forwards network traffic.Datacenter SDN architectures feature software-defined overlays or controllers that are abstracted from the underlying network hardware, offering intent-or policy-based management of the network as a whole. This results in a datacenter network that is better aligned with the needs of application workloads through automated (thereby faster) provisioning, programmatic network management, pervasive application-oriented visibility, and where needed, direct integration with cloud orchestration platforms
\subsection{SD-WAN}
\justifying
As the name states, software-defined wide-area networks use software to control the connectivity, management and services between data centers and remote branches or cloud instances. Like its bigger technology brother, software-defined networking, SD-WAN decouples the control plane from the data plane.
\subsection{Cloud networking}
\justifying
Cloud networking is a type of IT infrastructure in which some or all of an organization’s network capabilities and resources are hosted in a public or private cloud platform, managed in-house or by a service provider, and available on demand.\\
Companies can either use on-premises cloud networking resources to build a private cloud network or use cloud-based networking resources in the public cloud, or a hybrid cloud combination of both. These network resources can include virtual routers, firewalls, and bandwidth and network management software, with other tools and functions available as required. 
\subsection{Network visibility and analytics}
\justifying
Network visibility is how data is collected, aggregated, distributed and served to those monitoring and analytics tools. Network visibility is also about creating a stable foundation for your security infrastructure. Feeding these tools data they can’t use risks overloading them and degrading their performance.
\subsection{Cloud Native Network Functions}
\justifying
A cloud native network function (CNF) is a network function designed and implemented to run inside containers. CNFs inherit all cloud native architectural and operational principles including K8s lifecycle management, agility, resilience, and observability. CNFs place your physical (PNF) and virtual network functions (VNF) inside containers. You receive many of the VNF advantages discussed earlier. In addition, you are no longer burdened with VM software overhead. Containers do not require a guest OS or hypervisor, and you can quickly spin up/spin down container CNFs as needed.
\subsection{Intent-based networking (IBN)}
\justifying
Intent-based networking (IBN) is an emerging technology concept that aims to apply a deeper level of intelligence and intended state insights to networking. Ideally, these insights replace the manual processes of configuring networks and reacting to network issues.
\subsection{Intrinsic Security}
\justifying
Intrinsic security is a fundamentally different approach to securing your business. Rather, it is a strategy for leveraging your infrastructure and control points in new ways—in real time—across any app, cloud, or device so that you can shift from a reactive security posture to a position of strength.
\subsection{AI Security}
\justifying
AI security refers to tools and techniques that leverage artificial intelligence (AI) to autonomously identify and/or respond to potential cyber threats based on similar or previous activity. AI security tools are often used to identify “good” versus “bad” by comparing the behaviors of entities across an environment to those in a similar environment
\subsection{Internet of Autonomous things (IoAT)}
\justifying
From Wikipedia, the free encyclopedia
Jump to navigationJump to search
Autonomous things, abbreviated AuT, or the Internet of autonomous things, abbreviated as IoAT, is an emerging term for the technological developments that are expected to bring computers into the physical environment as autonomous entities without human direction, freely moving and interacting with humans and other objects
\subsection{WiFi 6}
\justifying
Wi-Fi 6 is the next generation of Wi-Fi. It’ll still do the same basic thing — connect you to the internet — just with a bunch of additional technologies to make that happen more efficiently, speeding up connections in the process.
\section{Industry}
\subsection{Predictive Maintenance}
\justifying
Predictive maintenance (PdM) is maintenance that monitors the performance and condition of equipment during normal operation to reduce the likelihood of failures. Also known as condition-based maintenance, predictive maintenance has been utilized in the industrial world since the 1990s.
\subsection{Additive Manufacturing}
\justifying
It is yet, another technological advancement made possible by the transition from analog to digital processes. In recent decades, communications, imaging, architecture and engineering have all undergone their own digital revolutions. Now, AM can bring digital flexibility and efficiency to manufacturing operations.
\subsection{Electronic Manufacturing Services (EMS)}
\justifying
Electronics manufacturing services (EMS) are those offered by companies that design, test, build, deliver, or give aftermarket help for electronic parts and assemblies. 
\subsection{Operationalizing Quality Assurance Operations QAOps}
\justifying
Put simply, QAOps means integrating software testing into the CI/CD pipeline, rather than performing software tests on a one-off basis. The core idea behind QAOps is to increase direct collaboration between developers and testing engineers instead of having them work in isolation.
\subsection{Acceptance Test Driven Development (ATDD)}
\justifying
Acceptance Test Driven Development (ATDD) aims to help a project team flesh out user stories into detailed Acceptance Tests that, when executed, will confirm whether the intended functionality exists.
\subsection{Digital twins}
\justifying
A digital twin is a digital representation of a physical object or system. The technology behind digital twins has expanded to include large items such as buildings, factories and even cities, and some have said people and processes can have digital twins, expanding the concept even further.
\subsection{Human augmentation}
\justifying
The field of human augmentation (sometimes referred to as “Human 2.0”) focuses on creating cognitive and physical improvements as an integral part of the human body. An example is using active control systems to create limb prosthetics with characteristics that can exceed the highest natural human performance.
\subsection{Hyper-Automation}
\justifying
It’s the extension of legacy business process automation beyond the confines of individual processes. By marrying AI tools with RPA, hyperautomation enables automation for virtually any repetitive task executed by business users.
It even takes it to the next level and automates the automation - dynamically discovering business processes and creating bots to automate them.
\end{document}